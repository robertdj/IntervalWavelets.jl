\documentclass[a4paper]{scrartcl}

% Text
\usepackage[utf8]{inputenc}
\usepackage[english]{babel}
\usepackage[T1]{fontenc}

% Math
\usepackage{amsmath, amssymb}
\usepackage{bm}
	\newcommand{\vv}[1]{\ensuremath{\bm{#1}}}
	\newcommand\lphi{\ensuremath{\phi^{\text{left}}}}
	\newcommand\lH{\ensuremath{H^{\text{left}}}}
	\newcommand\lh{\ensuremath{h^{\text{left}}}}

	\newcommand\rphi{\ensuremath{\phi^{\text{right}}}}
	\newcommand\rH{\ensuremath{H^{\text{right}}}}
	\newcommand\rh{\ensuremath{h^{\text{right}}}}

	\newcommand\R{\ensuremath{\mathbb{R}}}
	\newcommand\Z{\ensuremath{\mathbb{Z}}}
	\DeclareMathOperator\supp{supp}

\usepackage{graphicx}

\usepackage[colorlinks=true]{hyperref}

\usepackage{cleveref}

\usepackage{biblatex}
\bibliography{literature.bib}

\title{Daubechies wavelets}
\author{Robert Dahl Jacobsen}

\begin{document}

\maketitle

The purpose of the Julia package \href{https://github.com/robertdj/IntervalWavelets.jl}{IntervalWavelets.jl} is to compute ordinary Daubechies scaling functions (see e.g.\ \cite{Mallat:2009}) and the moment-preserving boundary scaling functions from \cite{Cohen:Daubechies:Vial:1993}.
We rely on the recursive approach of \cite{Strang:1989}.
In this short note I summarize these methods with all the details needed in the implementation.


\section{Interior scaling functions}
\label{sec:internal}

A Daubechies scaling function $\phi$ and associated wavelet $\psi$ with $p$ vanishing moments are defined by a filter $\{h_k\}_{k\in\Z}$.
The filter, scaling function and wavelet have supports of the same lengths and we know from \cite[Theorem 7.5]{Mallat:2009} that if $\supp \{h_k\}_k = [N_1, N_2]$, then 
\begin{equation*}
	\supp\phi = [N_1, N_2],
	\quad
	\supp\psi = \Bigl[\frac{N_1 - N_2 + 1}2, \frac{N_2 - N_1 + 1}2\Bigr].
\end{equation*}
It is customary to let $N_1 = 0$ and $N_2 = 2p - 1$.
However, when constructing the boundary scaling functions we have $N_1 = -p+1$ and $N_2 = p$.

The scaling function satisfies the dilation equation
\begin{equation}
	\label{eq:internal_scaling_function_definition}
	\phi(x) 
	= \sqrt2 \sum_{k=0}^{2p-1} h_k \phi_k(2x)
	= \sqrt2 \sum_{k=0}^{2p-1} h_k \phi(2x - k).
\end{equation}
For $p\geq2$, $\phi$ is continuous and hence zero at the endpoints of the support.
These properties allow us to compute $\phi$ at the integer values (in the support).
As an example, for $p = 3$:
\begin{align*}
	\phi(1) 
    & = \sqrt2 \bigl(h_1\phi(1) + h_0\phi(2)\bigr),
	\\
	\phi(2)
    & = \sqrt2 \bigl(h_4\phi(1) + h_3\phi(2) + h_2\phi(3) + h_1\phi(4)\bigr),
	\\
	\phi(3)
    & = \sqrt2 \bigl(h_5\phi(1) + h_4\phi(2) + h_3\phi(3) + h_2\phi(4)\bigr),
	\\
	\phi(4)
    & = \sqrt2 \bigl(h_5\phi(3) + h_4\phi(4)\bigr).
\end{align*}
In matrix form, we have an eigenvalue problem:
\begin{equation*}
	\begin{bmatrix}
		\phi(1) \\ \phi(2) \\ \phi(3) \\ \phi(4)
	\end{bmatrix}
	=
	\sqrt2
	\begin{bmatrix}
		h_1 & h_0 & 0 & 0
		\\
		h_3 & h_2 & h_1 & h_0
		\\
		h_5 & h_4 & h_3 & h_2
		\\
		0 & 0 & h_5 & h_4
	\end{bmatrix}
	\begin{bmatrix}
		\phi(1) \\ \phi(2) \\ \phi(3) \\ \phi(4)
	\end{bmatrix}.
\end{equation*}
For a general support $[N_1, N_2]$ the $(i,j)$'th entry is $\sqrt2 h_{2i-j+N_1}$ and the vector $\vv\phi = [\phi(n)]_{n=1}^4$ is an eigenvector of the eigenvalue 1.
This eigenspace is one-dimensional, so the only question is how to scale $\vv\phi$.
From e.g.\ \cite[page 69]{Cohen:Daubechies:Vial:1993} we know that
\begin{equation*}
	\sum_{k\in\mathbb{Z}} \phi(k)
	= \sum_{k=0}^{2p-1} \phi(k)
	= 1.
\end{equation*}

From the function values at the integers we can compute the function values at the half-integers using \eqref{eq:internal_scaling_function_definition}.
As an example,
\begin{equation*}
	\phi\Bigl(\frac32\Bigr)
    = \sqrt2 \bigl(h_0 \phi(3) + h_1 \phi(2) + h_2 \phi(1)\bigr).
\end{equation*}
This process can be repeated to recursively yield $\phi(k/2^R)$, for all integers $k$ and positive integers $R$.

Note that the filter $\{h_k\}_k$ defining the scaling function is not unique.
In \cref{fig:Daubechies4} is shown the usual, minimum-phase Daubechies 4 scaling function along with Daubechies 'symmlet'/linear phase scaling function used in \cref{sec:boundary_Daubechies} and \cite{Cohen:Daubechies:Vial:1993} -- see e.g.\ \cite[Section 7.2.3]{Mallat:2009}.

\begin{figure}
	\centering
	\includegraphics[scale=0.5]{interior}
	\caption{The usual minimum-phase Daubechies 4 scaling function (black) and the symmlet version (red).}
	\label{fig:Daubechies4}
\end{figure}


\section{Boundary scaling functions}
\label{sec:boundary_Daubechies}

The moment preserving Daubechies boundary scaling functions were introduced in \cite{Cohen:Daubechies:Vial:1993} and are also described in \cite{Mallat:2009} (albeit with some indexing errors).

An important difference between the internal and boundary scaling functions is that the left (right) boundary scaling functions are \emph{not} continuous at the left (right) endpoint of their support.

As in \cref{sec:internal}, the dilation equations defining the boundary scaling functions can yield function values at all dyadic rationals once we have the function values at the integers (in the support).
In the subsequent sections the focus is therefore on how to compute these functions at the integers.

The filters used for the boundary scaling functions are available at \url{https://services.math.duke.edu/~ingrid/publications/54.txt} and \url{http://numerical.recipes/contrib}.


\subsection{Left boundary scaling functions}

Let $p$ denote the number of vanishing moments and $\phi$ be the interior symmlet Daubechies scaling function associated with the wavelet with $p$ vanishing moments translated such that $\supp\phi = [-p+1, p]$.

We want a family of functions satisfying a multiresolution analysis on $L^2([0,\infty))$ or, equivalently, a dilation equation like \cref{eq:internal_scaling_function_definition}.
The starting point is $\{\phi_k\}_{k\geq 0}$.
The functions $\phi_k$ with $\supp\phi_k \subseteq [0,\infty)$ do not need any alteration.
But the $\phi_k$ with $\supp\phi_k \cap (-\infty,0) \neq \emptyset$ (i.e., with $0\leq k < p-1$) must be replaced with a corresponding $\lphi_k$.
It turns out that we should also replace $\phi_{p-1}$ with $\lphi_{p-1}$ in order to keep the number of vanishing moments even though $\supp\phi_{p-1} = [0,2p-1]$.
The boundary scaling functions are constructed such that $\supp\bigl(\lphi_k\bigr) = [0,p+k]$.

The relevant counterpart to the dilation equation \cref{eq:internal_scaling_function_definition} for interior scaling functions is
\begin{align}
	\lphi_k(x)
	& = \sqrt2 \sum_{l=0}^{p-1} \lH_{k,l} \lphi_l(2x) + \sqrt2 \sum_{m=p}^{p+2k} \lh_{k,m} \phi_m(2x)
	\nonumber
	\\
	& = \sqrt2 \sum_{l=0}^{p-1} \lH_{k,l} \lphi_l(2x) + \sqrt2 \sum_{m=p}^{p+2k} \lh_{k,m} \phi(2x-m),
	\label{eq:left_scaling_function_definition}
\end{align}
for $0\leq k\leq p-1$, where $(\lH_{k,l})$ and $(\lh_{k,m})$ are filter coefficients.

For $x \neq 0$ we make use of the compact support.
Consider e.g.\ the case $p=2$ (where is $\phi$ is supported on $[-1,2]$, $\lphi_0$ is supported on $[0,2]$ and $\lphi_1$ is supported on $[0,3]$):
\begin{align*}
	\lphi_0(1)
    & = \sqrt2 \bigl( \lH_{0,1} \lphi_1(2) + \lh_{0,2} \phi(0) \bigr),
	\\
	\lphi_1(1)
    & = \sqrt2 \bigl( \lH_{1,1} \lphi_1(2) + \lh_{1,2} \phi(0) \bigr),
	\\
	\lphi_1(2)
    & = \sqrt2 \bigl( \lh_{1,3} \phi(1) + \lh_{1,4} \phi(0) \bigr).
\end{align*}
From Section \ref{sec:internal} we know how to calculate the internal scaling function and by starting with the largest values we can also compute the boundary scaling function.

The function value $\lphi_k(0)$ require special treatment.
For $x = 0$ \eqref{eq:left_scaling_function_definition} becomes
\begin{equation*}
	\lphi_k(0)
    = \sqrt2 \sum_{l=0}^{p-1} \lH_{k,l} \lphi_l(0).
\end{equation*}
These function values are therefore an eigenvector of the matrix with $(i,j)$'th entry $[\sqrt2 \lH_{i,j}]$.
To find the proper normalization of this eigenvector we can do as follows:
Let $\lphi_\ell(0) = y_\ell$ for $0\leq \ell < p$.
Computing an eigenvector as described we have $\lphi_\ell(0) = z_\ell = c y_\ell$ for some $c \in \R$.
We know that the left boundary scaling functions are a basis capable of reconstructing polynomials.
In particular, there exists $a_0, \ldots, a_{p-1}$ such that for all $x \in [0, 1]$ we have
\begin{equation*}
    \sum_{\ell=0}^{p-1} a_\ell \lphi_\ell(x) = 1.
\end{equation*}
On this interval all the interior scaling functions are zero and we only need the left boundary scaling functions.
The coefficients can be determined by choosing $p$ dyadic rationals $x_0, \ldots, x_{p-1}$ with $0 < x_k \leq 1$ and solving the linear equations
\begin{equation*}
    \sum_{\ell=0}^{p-1} a_\ell \lphi_\ell(x_k) = 1, \quad 0\leq k< p.
\end{equation*}
In order to have have the $p$ function values $\lphi_\ell(x_k)$ the iterative refinement must proceed until the resolution of the dyadic rationals is at least $R = \lceil \log_2 p \rceil$.
This gives us the constant $c$:
\begin{equation*}
    1 = \sum_{\ell=0}^{p-1} a_\ell \lphi_\ell(0) 
    \quad\Rightarrow\quad
    c = \sum_{\ell=0}^{p-1} a_\ell z_\ell. 
\end{equation*}


The four boundary scaling functions related to four vanishing moments are seen in \cref{fig:left_Daubechies4}.
There is a large resemblance between $\lphi_3$ and the symmlet scaling function in \cref{fig:Daubechies4} (here denoted $\phi_4$).

\begin{figure}
	\centering
	\includegraphics[scale=0.5]{left}
	\caption{The left boundary scaling function with 4 vanishing moments.}
	\label{fig:left_Daubechies4}
\end{figure}


\subsection{Right boundary scaling functions}

Let again $\phi$ denote the interior symmlet Daubechies scaling function and $p$ denote the number of vanishing moments of the associated wavelet.
The idea for the right boundary scaling functions is the same as for the the left:
We want a multiresolution analysis on $L^2((-\infty,0])$ by modifiying the interior scaling functions.
The $\phi_k$ with $\supp\phi_k \subset (-\infty,0)$ are unaltered, but those with $\supp\phi_k \cap [0, \infty) \neq \emptyset$ are replaced by a corresponding $\rphi_k$.
In conclusion, for $k=0,\ldots,p-1$, the right boundary scaling functions satisfies the dilation equations
\begin{equation}
	\label{eq:right_scaling_function_definition}
	\rphi_k(x)
	= \sqrt2 \sum_{l=0}^{p-1} \rH_{k,l} \rphi_l(2x) + \sqrt2 \sum_{m=p}^{p+2k} \rh_{k,m} \phi(2x+m+1),
\end{equation}
where $(\rH_{k,l})$ and $(\rh_{k,m})$ are filter coefficients.
The support of $\rphi_k$ is $[-p-k,0]$.

The four right boundary scaling functions related to four vanishing moments are seen in \cref{fig:right_Daubechies4}.

\begin{figure}
	\centering
	\includegraphics[scale=0.5]{right}
	\caption{The right boundary scaling function with 4 vanishing moments.}
	\label{fig:right_Daubechies4}
\end{figure}


\printbibliography

\end{document}

